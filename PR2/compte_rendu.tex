\documentclass[a4paper,12pt]{article}

\usepackage[utf8]{inputenc}
\usepackage{graphicx}
\usepackage{amsmath}
\usepackage{multicol}
\usepackage{breqn}
\usepackage{xcolor}
\newcommand\tab[1][1cm]{\hspace*{#1}}
\definecolor{gray0}{RGB}{235,235,235}
\definecolor{gray1}{RGB}{220,220,220}
\definecolor{gray2}{RGB}{205,205,205}

\begin{document}

	\title{\textbf{Compte rendu Projet 2 - SE201}}
	\date{October 7, 2016}
	\maketitle

	\Large{\textbf{3 - Pipelining}}\\\break
    \small
    \underline{3.1}\\
    \setlength{\parindent}{5ex}\par
	En observant les instructions, on remarque un data hazard lors de l'execution de l'instruction 3.
	En effet, le store (instrution 3) n�cessite la valeur du registre 29, mais celle-ci n'est pas encore � jour.
	L'instruction pr�c�dent qui �crit dans ce registre n'est pas encore termin�e.\\
	Au cycle 17, la valeur de r29 n'est pas � jour mais elle devrait �tre charg�e par le store qui est en ID.
	Au cycle 18, l'instruction 3 passe directement � l'execution sans avoir attentdu de pouvoir lire la valeur du registre 29.
	Il y a donc bien forwarding.
	((((((((((-------->  INCLUDE 1.PNG  <---------))))))))))))
	\\\break
    \underline{3.2}\\
    \par
	Premi�re exemple : \\
	((((((((((-------->  INCLUDE 2.PNG  <---------))))))))))))
	Dans le cas d'un jump (jr) ou d'une condition (beqz, bnez) � cause une "misprediction" le processeur va "flush" les instructions qui avaient �t� commenc�. 
	Celles-ci ne se finiront jamais.  
	Second exemple : \\
	((((((((((-------->  INCLUDE 3.PNG  <---------))))))))))))
	\\\break
\underline{3.3}\\
    \par

\end{document}
